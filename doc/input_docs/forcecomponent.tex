\section{FORCECOMPONENT}
\label{FORCECOMPONENT}
\begin{ipifield}{}%
{The class that deals with how each forcefield contributes to the overall potential, force and virial calculation.}%
{}%
{\ipiitem{forcefield}%
{Mandatory. The name of the forcefield this force is referring to.}%
{default: `'; data type: string; }%
\ipiitem{fd\_epsilon}%
{The finite displacement to be used for calculaing the Suzuki-Chin contribution of the force. If the value is negative, a centered finite-difference scheme will be used. [in bohr]}%
{default:  -0.001 ; data type: float; }%
\ipiitem{name}%
{An optional name to refer to this force component.}%
{default: `'; data type: string; }%
\ipiitem{weight}%
{A scaling factor for this forcefield, to be applied before adding the force calculated by this forcefield to the total force.}%
{default:  1.0 ; data type: float; }%
\ipiitem{nbeads}%
{If the forcefield is to be evaluated on a contracted ring polymer, this gives the number of beads that are used. If not specified, the forcefield will be evaluated on the full ring polymer.}%
{default:  0 ; data type: integer; }%
}
\begin{ipifield}{mts\_weights}%
{The weight of force in each mts level starting from outer.}%
{dimension: force; default: 
      [1.]; data type: float; }%
{\ipiitem{units}%
{The units the input data is given in.}%
{default: `automatic'; data type: string; }%
\ipiitem{shape}%
{The shape of the array.}%
{default:  (1,) ; data type: tuple; }%
\ipiitem{mode}%
{If 'mode' is 'manual', then the array is read from the content of 'cell' takes a 9-elements vector containing the cell matrix (row-major). If 'mode' is 'abcABC', then 'cell' takes an array of 6 floats, the first three being the length of the sides of the system parallelopiped, and the last three being the angles (in degrees) between those sides. Angle A corresponds to the angle between sides b and c, and so on for B and C. If mode is 'abc', then this is the same as for 'abcABC', but the cell is assumed to be orthorhombic. 'pdb' and 'chk' read the cell from a PDB or a checkpoint file, respectively.}%
{default: `manual'; data type: string; options: `manual', `file'; }%
}
\end{ipifield}
\end{ipifield}
