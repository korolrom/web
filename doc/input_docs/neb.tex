\section{NEB}
\label{NEB}
\begin{ipifield}{}%
{Contains the required parameters for performing nudged elastic band (NEB) calculations}%
{}%
{\ipiitem{mode}%
{The geometry optimization algorithm to be used}%
{default: `lbfgs'; data type: string; options: `sd', `cg', `bfgs', `lbfgs'; }%
}
\begin{ipifield}{biggest\_step}%
{The maximum step size for (L)-BFGS line minimizations.}%
{default:  100.0 ; data type: float; }%
{}
\end{ipifield}
\begin{ipifield}{scale\_lbfgs}%
{Scale choice for the initial hessian.
                                            0 identity.
                                            1 Use first member of position/gradient list.
                                            2 Use last  member of position/gradient list.}%
{default:  2 ; data type: integer; }%
{}
\end{ipifield}
\begin{ipifield}{corrections\_lbfgs}%
{The number of past vectors to store for L-BFGS.}%
{default:  5 ; data type: integer; }%
{}
\end{ipifield}
\begin{ipifield}{tolerances}%
{Generic input value}%
{}%
{}
\end{ipifield}
\begin{ipifield}{glist\_lbfgs}%
{List of previous gradient differences for L-BFGS, if known.}%
{default:  [ ] ; data type: float; }%
{\ipiitem{shape}%
{The shape of the array.}%
{default:  (0,) ; data type: tuple; }%
\ipiitem{mode}%
{If 'mode' is 'manual', then the array is read from the content of 'cell' takes a 9-elements vector containing the cell matrix (row-major). If 'mode' is 'abcABC', then 'cell' takes an array of 6 floats, the first three being the length of the sides of the system parallelopiped, and the last three being the angles (in degrees) between those sides. Angle A corresponds to the angle between sides b and c, and so on for B and C. If mode is 'abc', then this is the same as for 'abcABC', but the cell is assumed to be orthorhombic. 'pdb' and 'chk' read the cell from a PDB or a checkpoint file, respectively.}%
{default: `manual'; data type: string; options: `manual', `file'; }%
}
\end{ipifield}
\begin{ipifield}{spring}%
{Uniform or variable spring constants along the elastic band}%
{}%
{}
\end{ipifield}
\begin{ipifield}{old\_direction}%
{The previous direction.}%
{default:  [ ] ; data type: float; }%
{\ipiitem{shape}%
{The shape of the array.}%
{default:  (0,) ; data type: tuple; }%
\ipiitem{mode}%
{If 'mode' is 'manual', then the array is read from the content of 'cell' takes a 9-elements vector containing the cell matrix (row-major). If 'mode' is 'abcABC', then 'cell' takes an array of 6 floats, the first three being the length of the sides of the system parallelopiped, and the last three being the angles (in degrees) between those sides. Angle A corresponds to the angle between sides b and c, and so on for B and C. If mode is 'abc', then this is the same as for 'abcABC', but the cell is assumed to be orthorhombic. 'pdb' and 'chk' read the cell from a PDB or a checkpoint file, respectively.}%
{default: `manual'; data type: string; options: `manual', `file'; }%
}
\end{ipifield}
\begin{ipifield}{old\_force}%
{The previous force in an optimization step.}%
{dimension: force; default:  [ ] ; data type: float; }%
{\ipiitem{units}%
{The units the input data is given in.}%
{default: `automatic'; data type: string; }%
\ipiitem{shape}%
{The shape of the array.}%
{default:  (0,) ; data type: tuple; }%
\ipiitem{mode}%
{If 'mode' is 'manual', then the array is read from the content of 'cell' takes a 9-elements vector containing the cell matrix (row-major). If 'mode' is 'abcABC', then 'cell' takes an array of 6 floats, the first three being the length of the sides of the system parallelopiped, and the last three being the angles (in degrees) between those sides. Angle A corresponds to the angle between sides b and c, and so on for B and C. If mode is 'abc', then this is the same as for 'abcABC', but the cell is assumed to be orthorhombic. 'pdb' and 'chk' read the cell from a PDB or a checkpoint file, respectively.}%
{default: `manual'; data type: string; options: `manual', `file'; }%
}
\end{ipifield}
\begin{ipifield}{qlist\_lbfgs}%
{List of previous position differences for L-BFGS, if known.}%
{default:  [ ] ; data type: float; }%
{\ipiitem{shape}%
{The shape of the array.}%
{default:  (0,) ; data type: tuple; }%
\ipiitem{mode}%
{If 'mode' is 'manual', then the array is read from the content of 'cell' takes a 9-elements vector containing the cell matrix (row-major). If 'mode' is 'abcABC', then 'cell' takes an array of 6 floats, the first three being the length of the sides of the system parallelopiped, and the last three being the angles (in degrees) between those sides. Angle A corresponds to the angle between sides b and c, and so on for B and C. If mode is 'abc', then this is the same as for 'abcABC', but the cell is assumed to be orthorhombic. 'pdb' and 'chk' read the cell from a PDB or a checkpoint file, respectively.}%
{default: `manual'; data type: string; options: `manual', `file'; }%
}
\end{ipifield}
\begin{ipifield}{invhessian\_bfgs}%
{Approximate inverse Hessian for BFGS, if known.}%
{default:  [ ] ; data type: float; }%
{\ipiitem{shape}%
{The shape of the array.}%
{default:  (0, 0) ; data type: tuple; }%
\ipiitem{mode}%
{If 'mode' is 'manual', then the array is read from the content of 'cell' takes a 9-elements vector containing the cell matrix (row-major). If 'mode' is 'abcABC', then 'cell' takes an array of 6 floats, the first three being the length of the sides of the system parallelopiped, and the last three being the angles (in degrees) between those sides. Angle A corresponds to the angle between sides b and c, and so on for B and C. If mode is 'abc', then this is the same as for 'abcABC', but the cell is assumed to be orthorhombic. 'pdb' and 'chk' read the cell from a PDB or a checkpoint file, respectively.}%
{default: `manual'; data type: string; options: `manual', `file'; }%
}
\end{ipifield}
\begin{ipifield}{climb}%
{Use climbing image NEB}%
{default:  False ; data type: boolean; }%
{}
\end{ipifield}
\begin{ipifield}{endpoints}%
{Geometry optimization of endpoints}%
{}%
{}
\end{ipifield}
\begin{ipifield}{ls\_options}%
{Options for line search methods. Includes:
                              tolerance: stopping tolerance for the search,
                              grad\_tolerance: stopping tolerance on gradient for
                              BFGS line search,
                              iter: the maximum number of iterations,
                              step: initial step for bracketing,
                              adaptive: whether to update initial step.
                              }%
{}%
{}
\end{ipifield}
\end{ipifield}
