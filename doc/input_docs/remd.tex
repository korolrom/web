\section{REMD}
\label{REMD}
\begin{ipifield}{}%
{Replica Exchange}%
{}%
{}
\begin{ipifield}{stride}%
{Every how often to try exchanges (on average).}%
{default:  1.0 ; data type: float; }%
{}
\end{ipifield}
\begin{ipifield}{krescale}%
{Rescale kinetic energy upon exchanges.}%
{default:  True ; data type: boolean; }%
{}
\end{ipifield}
\begin{ipifield}{repindex}%
{List of current indices of the replicas compared to the starting indices}%
{default:  [ ] ; data type: integer; }%
{\ipiitem{shape}%
{The shape of the array.}%
{default:  (0,) ; data type: tuple; }%
\ipiitem{mode}%
{If 'mode' is 'manual', then the array is read from the content of 'cell' takes a 9-elements vector containing the cell matrix (row-major). If 'mode' is 'abcABC', then 'cell' takes an array of 6 floats, the first three being the length of the sides of the system parallelopiped, and the last three being the angles (in degrees) between those sides. Angle A corresponds to the angle between sides b and c, and so on for B and C. If mode is 'abc', then this is the same as for 'abcABC', but the cell is assumed to be orthorhombic. 'pdb' and 'chk' read the cell from a PDB or a checkpoint file, respectively.}%
{default: `manual'; data type: string; options: `manual', `file'; }%
}
\end{ipifield}
\begin{ipifield}{swapfile}%
{File to keep track of replica exchanges}%
{default: `PARATEMP'; data type: string; }%
{}
\end{ipifield}
\end{ipifield}
