\section{PHONONS}
\label{PHONONS}
\begin{ipifield}{}%
{Fill in.}%
{}%
{\ipiitem{mode}%
{The algorithm to be used: finite differences (fd), normal modes finite differences (nmfd), and energy-scaled normal mode finite differences (enmfd).}%
{default: `fd'; data type: string; options: `fd', `nmfd', `enmfd'; }%
}
\begin{ipifield}{asr}%
{Removes the zero frequency vibrational modes depending on the symmerty of the system.}%
{default: `none'; data type: string; options: `none', `poly', `lin', `crystal'; }%
{}
\end{ipifield}
\begin{ipifield}{prefix}%
{Prefix of the output files.}%
{default: `PHONONS'; data type: string; }%
{}
\end{ipifield}
\begin{ipifield}{dynmat}%
{Portion of the dynamical matrix known up to now.}%
{default:  [ ] ; data type: float; }%
{\ipiitem{shape}%
{The shape of the array.}%
{default:  (0,) ; data type: tuple; }%
\ipiitem{mode}%
{If 'mode' is 'manual', then the array is read from the content of 'cell' takes a 9-elements vector containing the cell matrix (row-major). If 'mode' is 'abcABC', then 'cell' takes an array of 6 floats, the first three being the length of the sides of the system parallelopiped, and the last three being the angles (in degrees) between those sides. Angle A corresponds to the angle between sides b and c, and so on for B and C. If mode is 'abc', then this is the same as for 'abcABC', but the cell is assumed to be orthorhombic. 'pdb' and 'chk' read the cell from a PDB or a checkpoint file, respectively.}%
{default: `manual'; data type: string; options: `manual', `file'; }%
}
\end{ipifield}
\begin{ipifield}{output\_shift}%
{Shift by the dynamical matrix diagonally before outputting.}%
{default:  0.0 ; data type: float; }%
{}
\end{ipifield}
\begin{ipifield}{energy\_shift}%
{The finite displacement in energy used to compute derivative of force.}%
{default:  0.0 ; data type: float; }%
{}
\end{ipifield}
\begin{ipifield}{refdynmat}%
{Portion of the refined dynamical matrix known up to now.}%
{default:  [ ] ; data type: float; }%
{\ipiitem{shape}%
{The shape of the array.}%
{default:  (0,) ; data type: tuple; }%
\ipiitem{mode}%
{If 'mode' is 'manual', then the array is read from the content of 'cell' takes a 9-elements vector containing the cell matrix (row-major). If 'mode' is 'abcABC', then 'cell' takes an array of 6 floats, the first three being the length of the sides of the system parallelopiped, and the last three being the angles (in degrees) between those sides. Angle A corresponds to the angle between sides b and c, and so on for B and C. If mode is 'abc', then this is the same as for 'abcABC', but the cell is assumed to be orthorhombic. 'pdb' and 'chk' read the cell from a PDB or a checkpoint file, respectively.}%
{default: `manual'; data type: string; options: `manual', `file'; }%
}
\end{ipifield}
\begin{ipifield}{pos\_shift}%
{The finite displacement in position used to compute derivative of force.}%
{default:  0.01 ; data type: float; }%
{}
\end{ipifield}
\end{ipifield}
